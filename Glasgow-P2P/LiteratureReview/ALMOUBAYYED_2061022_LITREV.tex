\documentclass[12pt,helvetica,a4paper,final]{iopart}
\usepackage{iopams}  
\linespread{1.5}
\usepackage[tmargin=4cm, lmargin=3cm, rmargin=3cm]{geometry}
\usepackage[breaklinks=true,colorlinks=true,linkcolor=blue,urlcolor=blue,citecolor=blue]{hyperref}
\usepackage{caption,subcaption}
\usepackage{apacite} 
\bibliographystyle{apacite}

\begin{document}
\maketitle

\title[Peer-Teaching in Higher Education]{Pear-Teaching in Higher Education: From the Perspective of Student-Tutors}
\smallskip
\author{Husni Almoubayyed}
\address{School of Physics and Astronomy, University of Glasgow}
\eads{\mailto{2061022a@student.glasgow.ac.uk}}
\section{Introduction to Peer Teaching}

The classical methods of classroom teaching have existed for a long time, and albeit working reasonably well, teachers aim to subsume new innovative methods that increase the productivity of classroom teaching. One of these methods is peer-teaching~\cite{lausanne}, also known as peer-assisted tutoring, which has been increasing in popularity.


Peer-teaching is defined as the teaching process where a student assists the learning of, or acts as the tutor to another student in the same or a similar status academically (or assist the learning of a small group of students in the same status)~\cite{theory}.

Peer-teaching has many advantages for both the tutor and the tutee, but it is ill-advised to consider it a panacea or a cheaper alternative for traditional teaching~\cite{dundee}, and only incorporating both methods can add value to the classroom teaching.

This report aims to discuss the benefits, limitations, and pitfalls of peer-teaching, focusing on that from the perspective of the student-tutors. 
\section{Impact of Peer-Assisted Teaching on Student-Tutors}

While possibly the major goal of peer teaching is to improve the academic education of the tutees, it largely affects the student-tutors as well, for example, peer-teaching can augment the academic gain of the tutors through teaching the material and the repetitive exposure to it, answering the tutee's questions, as well as improving the tutor's structural learning skills by organising the material and developing work habits~\cite{theory}. It also refines the tutors' cognitive skills, and the tutor's sense of responsibility, by acting as a role model, facilitator and information provider~\cite{medical}.

\par{}
Peer-teachers, however, should not substitute professors, as they might not have a full comprehension of the material and might not be able to deliver the information as well while teaching, therefore, the tutors must be monitored closely, especially those less highly achieving than others~\cite{theory}.

\par{}
Depending on the certain responsibilities of student-tutors, peer teaching improves the accountability and assessment skills for the tutor~\cite{colvin}. Peer assessment in particular can help tutors improve teamwork skills and active learning, develop negotiation and verbal communication skills, giving and accepting criticism and
accepting or rejecting suggestions. On the other hand, student-tutors might be at a discomfort of having the responsibility of assessing their peers, and tutees may not agree with the assessments given by their peer-tutors, or think of it less as a truthful reflection of the tutees' abilities in comparison to an assessment coming from a professor~\cite{keith}.

\par{}
The tutors can also improve their organisational skills by preparing and organising the content they are responsible for towards the tutees, which helps them in turn form a comprehensive understanding of the material by persistent retention. Peer-teaching has been found to increase the creativity of student-tutors. Furthermore, the student-tutors both receive and give a considerable amount of feedback because of the low $\frac{tutor}{tutee}$ ratio, which also amplifies the individualisation in the teaching and learning processes between the tutor and tutee~\cite{theory}. Furthermore, peer-teaching improves collegiality in the department, and the academic practices in the department both from being observed and observing other student-tutors~\cite{double}, as well as the relationships between tutors and tutees~\cite{longauthorlist}.

\par{}
Socially, the tutors benefits from the high regard and academic status that accompanies the responsibilities of peer-teaching, since usually only more-accomplished students can become peer-tutors, or at least do well at it. This can also act as a motive for the student-tutors to take on this form of teaching, along with other motives such as academic gain, interactive learning, and other moral motives that inspire the student-tutors to assist other students, especially ones that are not as highly-achieving and privileged as others~\cite{theory}.

\par{}
Peer-teaching, however, requires a prerequisite set of social skills, such as empathy, understanding, and communication skills to help the tutors achieve their responsibilities fittingly and appropriately. If the peer tutors do not have these skills to the same level of the professors, which they usually do not, they might not be able to realise their roles to the same effect~\cite{theory}.

\par{}
In addition, the student-tutors' teaching and presentation skills, which is a large part of the academic practice, can further benefit from teaching~\cite{stem}, such skills are useful in the future career and development of the academic personality of the student-tutors. The tutors also learn persistence, time-management, punctuality, taking on a leading role rather than a peer role~\cite{theory},  justifying a position, increasing attention and motivation~\cite{dundee}, that can also be useful for their future careers.

\par{}
Some researches~\cite{learntwice} argue that education research and the analysis of peer teaching is still at a rudimentary stage and that teachers must be cautious (but not completely discouraged) when it comes to integrating peer-teaching into the classroom. It should also not be a substitute to traditional teaching.

\section{Conclusion}
After centuries of trying the traditional classroom teaching method, more methods are being tested to improve the teaching and learning experience. Peer-teaching is one of these methods, which allows students of the same, or of a close academic level to tutor their peers. Peer-teaching aims to help assist the teachers and students with the learning process, to ensure a better quality of education for those students. Several benefits for peer-tutors however, have been achieved, with peer-teaching. These include a higher understanding of the material by the tutors, improving organisational and communication skills through this interactive method, the sense of accountability and responsibility, as well as empathy and understanding, social skills, presentation skills, and a sense of satisfaction that accompanies helping other students. There are, however, certain pitfalls of peer-teaching, such as the potential inadequate understanding of the student-tutor for the material compared to that of a professor, the possible lack of empathy and understanding towards the tutees. Student-tutors may also be put in an uncomfortable situation if they are responsible for assessing their peers, and might become a target for their peers' criticism. Therefore, teachers should be careful, but not discouraged when integrating peer-teaching into the classroom.
%\section*{References}% !TEX root =  
%\begin{flushleft}
{\small
\bibliography{biblio}}
%\end{flushleft}
\end{document}